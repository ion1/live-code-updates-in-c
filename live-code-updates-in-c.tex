\documentclass{beamer}
\usetheme{default}

\usepackage{xltxtra,xunicode}

\usepackage{polyglossia}
\setdefaultlanguage{english}

\usepackage{minted}

\setlength{\parskip}{\medskipamount}
\setlength{\parindent}{0pt}

\title{Live code updates in C}
\subtitle{What we might learn from Erlang}
\author{Johan Kiviniemi}

\begin{document}

\begin{frame}[plain]
  \titlepage
\end{frame}

\begin{frame}{Introduction}
  Erlang has a built-in facility for easy and robust live code updates.

  I’ll describe what Erlang is doing first, and then contemplate how we could
  imitate that in a single-threaded C program based on a libnih main loop.

  Erlang/OTP comes with abstraction layers one typically uses, but I’ll use
  low-level Erlang in the examples to make everything that happens explicit.
\end{frame}

\begin{frame}[fragile]{A simple server}
  An Erlang module implementing a server pattern might behave as follows:

  \inputminted{erlang}{pingpong-00-example.erl-sh}
\end{frame}

\begin{frame}[fragile]
  The implementation might look as follows:

  \begin{figure}[H]
    \inputminted[lastline=13]{erlang}{pingpong-00.erl}
    \caption{\texttt{pingpong.erl}: The API}
  \end{figure}
\end{frame}

\begin{frame}[fragile]
  \begin{figure}[H]
    \inputminted[firstline=15]{erlang}{pingpong-00.erl}
    \caption{\texttt{pingpong.erl}: The server}
  \end{figure}
\end{frame}

\begin{frame}[fragile]{The code explained}
  Let’s go over the code.

  \inputminted[firstline=6,lastline=7]{erlang}{pingpong-00.erl}

  The \verb|start| function takes an arbitrary name to identify an instance and
  spawns a process – Erlang jargon for an extremely lightweight thread – using
  \verb|spawn_link|, returning the new process ID.

  The parameters to \verb|spawn_link|:

  \begin{itemize}
    \item \verb|?MODULE| refers to the current module.

    \item \verb|loop| refers to the name of the main loop function for the
    process (within \verb|?MODULE|).

    \item The third parameter provides a list of parameters to which
    \verb|?MODULE:loop| is applied to. In this case, just a single parameter,
    the tuple \verb|{pingpong, Name}| where \verb|Name| came from the caller.
    This represents the initial state for the loop.
  \end{itemize}
\end{frame}

\begin{frame}[fragile]
  \inputminted[firstline=9,lastline=9]{erlang}{pingpong-00.erl}

  The \verb|ping| function is evaluated by a caller, not the server spawned by
  \verb|start|. It sends a message to the server and returns the result.
  \verb|ping| is applied to the pid returned by start.

  \inputminted[firstline=10,lastline=10]{erlang}{pingpong-00.erl}

  \verb|Pid ! Message| is Erlang’s internal syntax for IPC. Each process has a
  mailbox to which any process can add messages by means of the \verb|!|
  operator.

  In this case, the message \verb|{ping, self()}| is added to the server’s
  mailbox. \verb|self()| returns the pid of the current process. We add that to
  the message so that the server knows where to send the reply. (This is the
  kind of stuff the abstraction modules would handle normally.)
\end{frame}

\begin{frame}[fragile]
  \inputminted[firstline=11,lastline=13]{erlang}{pingpong-00.erl}

  The \verb|receive| construct takes a list of patterns to match against the
  mailbox.  The first message in the mailbox to match one of the patterns is
  removed and handled by the handler. If no messages match, \verb|receive|
  blocks until a matching message is received or an optional timeout occurs.
\end{frame}

\begin{frame}[fragile]
  \inputminted[firstline=17,lastline=17]{erlang}{pingpong-00.erl}

  The \verb|loop| function implements the server. It’s run in a separate
  process created by the \verb|spawn_link| call.

  The strange-looking argument matcher simply means that the function
  definition only matches a tuple parameter that looks like
  \verb|{pingpong, SOMETHING}|, where the entire tuple is assigned to the
  variable \verb|State| and the \emph{something} is assigned to the variable
  \verb|Name|.
\end{frame}

\begin{frame}[fragile]
  The function body consists of just a \verb|receive| construct.

  \inputminted[firstline=18,lastline=21]{erlang}{pingpong-00.erl}

  When the message \verb|{ping, SOMETHING}| is received, the \emph{something}
  is assigned to \verb|Sender|, and the message \verb|{pong, self(), {Name}}|
  is sent to it. \verb|Name| came from the state parameter, and ultimately from
  the original parameter to \verb|start|.

  To get back to the beginning of the loop in order to handle the next incoming
  message, we call ourselves again (\verb|loop(State)|) and the tail recursion
  optimization converts that to not much more than a jump.
\end{frame}

\begin{frame}[fragile]
  It’s important to note a few things:

  \begin{itemize}
    \item The entire state of the process is on the \emph{stack}.

    \item Execution proceeds back to the beginning of the loop by using a tail
    call.

    \item The new state for the process is given as a parameter to the tail
    call.  (So far we haven’t modified the state.)

    \item \verb|loop(State)| refers to the \verb|loop| function in the
    \emph{current} version of the module, whereas \verb|?MODULE:loop(State)|
    would refer to the \verb|loop| function in the \emph{newest} version of the
    module. Those two can differ when a new version of the module has been
    loaded.
  \end{itemize}
\end{frame}

\begin{frame}[fragile]
  Now that we’ve got the basics out of the way, let’s make the module friendly
  to runtime code upgrades.
\end{frame}

\end{document}
